\documentclass[11pt]{article}

\input strowpreprint

\newcommand{\kc}{\textsf{kCARTA}\xspace}

\lhead{\textsf{\textbf{DRAFT}}}

\title{kCARTA : A Fast Pseudo Line by Line Radiative Transfer Algorithm with
                Analytic Jacobians, Scattering and NonLocal Thermodynamic 
                Equilibrium Radiative Transfer}

\author{Sergio De Souza-Machado, L. Larrabee Strow,\\
       Howard E. Motteler, and Scott E. Hannon\\
       University of Maryland Baltimore County, Baltimore, MD 21250 USA}

\begin{document}

\maketitle

\begin{abstract}
  
  Using a custom line-by-line spectroscopic code, a radiative transfer package
  using a monochromatic \textsf{SVD} compressed database has been developed. 
  We use one case study to illustrate use of the package, determining the 
  sizes of the various contributing derivative terms (Planck radiance versus 
  weighting functions) in the temperature Jacobians. 
\end{abstract}

\section{Introduction}

\subsection{Spectroscopic line-by-line codes}

When using the spectra produced by spectroscopy codes in radiative transfer
algorithms, the accuracy of these codes is especially important.
For example, line mixing and far-wing effects have to be correctly and
accurately incorporated into lineshapes, as atmospheric parameters 
such as temperature and humidity as a function of atmospheric height can be
retrieved using information in spectral regions dominated by these effects. 
While these codes can be very accurate, when compared to experimentally 
obtained laboratory spectra, they are usually too slow for practical 
``on-the-fly'' applications.

One approach to implement line mixing or far wing effects is to add on a 
continuum, or to use a chi function to  multiply a Voigt or Lorentz 
lineshape. While an empirical continuum is easy to implement in a 
line-by-line code, our recent
work has shown that for water, non-Lorentz line shapes must be used
close to the line centers, a complicating factor in traditional
implementations of line-by-line codes. Similarly, line mixing effects
in between \cd lines in the 4.3 and 15 $\mu$m regions are also complicated.
These problems are not impossible to deal with, but they significantly 
complicate codes that are already difficult to develop and maintain.

The upcoming generation of nadir viewing satellite instruments for
remote sensing of atmospheric temperature and humidity profiles will
require accurate forward models for radiative transfer.  This is a
rather demanding task since these new instruments, such as the
Atmospheric InfraRed Sounder (AIRS)\cite{air:91}, have thousands of
low-noise spectral channels throughout the infrared.  Physical
retrieval algorithms using these high-resolution radiances require the
rapid computation of accurate radiances at the instrument spectral
resolution.  This is usually achieved by parameterizing effective
layer transmittances at the instrument spectral resolution as a
function of the temperature and constituent profile, for a wide 
variety of atmospheric profiles. To do this, monochromatic transmittances 
need to be generated for the large number of atmospheric profiles. This 
is a cumbersome task with line-by-line codes, given that spectral 
resolutions on the order of 0.0005 \wn\ are required over almost 2500 \wn.  
Although reducing this
tedious computation was the motivating force behind the development of
the work presented here, there are many other applications that
require copious, and slow, line-by-line computations.

\subsection{Combining spectroscopy and radiative transfer algorithms}
The problem of code maintenance is compounded if one wants to combine a 
spectroscopy code with a detailed radiative transfer code. For example, there
could be updates to the computation machinery required for the spectroscopy of
any one of the gases, such as line mixing for \cd, or the release of new
line parameters in the \textsf{HITRAN} database. Another update could be 
including and refining a radiative transfer algorithm with the code.

For the reasons presented above, we developed a highly compressed precomputed 
database of monochromatic atmospheric optical depths that is accurate, 
relatively small, and easy to use.  Generation of monochromatic transmittances
from this database for an arbitrary Earth atmosphere temperature, pressure 
and gas amount profile is orders of magnitude faster than using a 
line-by-line code. We call this the kCompressed Database. For each gas that
is radiatively important in the infrared region, the database 
consists of binary files in blocks of 25 $cm^{-1}$. While the ``line-by-line''
code that is used to generate the spectroscopic database can be slow and 
complicated, these details do not trouble the user of the kCompressed 
Database, as all that is required is an algorithm to uncompress the database 
for a given profile. To update the spectroscopy in a selected wavenumber 
region for a specified gas is now trivial, as all that is required is the 
relevant updated  file in the kCompressed database. 

In addition to the uncompression algorithm that computes spectra for the
desired atmospheric profile, we have developed clear sky radiative transfer 
algorithms for both a downlooking and an uplooking instrument. Radiances 
computed using the compressed database are as accurate as those computed with
a line-by-line code since our compression procedure introduces errors well 
below spectroscopy errors. Although this method {\em is} much slower than 
fast forward models based on effective convolved transmittances, it is much 
more accurate. 

Twice daily global coverage of the Earth by the AIRS (Atmospheric
InfraRed Sounder) instrument, has provided 
us with radiometrically accurate data throughout the infrared
region. AIRS is a high resolution infrared instrument launched on
board NASA's AQUA platform in May 2002, with 2378 channels spanning
the infrared region from 650 to 2700 \wn. The resolution of the
channels is given by $\nu/1200$, giving FWHM of
approximately 0.5 \wn and 2 \wn at 650,2400 \wn respectively. The Fast
Forward Clear Sky Algorithm for AIRS is developed from \kc \cite{str:02*2}.
The radiometric accuracy of AIRS is better than 0.5 K in the
10 \um and 3.7 \um window regions \cite{aum:02*2}. This high
resolution spectrally accurate data has enabled us to fine tune some
of the spectroscopy used in the kCompressed Database, such as the 4.3
micron \cd $R$ branchead, and the water
continuum coefficients in the 6.7 micron water vapor region. Additionally, 
the instrument has provided us with the first high resolution nadir data of 
Non Local Thermodynamic Equilibrium (NLTE) in the 4 micron \cd band. Above 
45 km, solar pumping during the day
significantly alters the vibrational temperatures from the local kinetic 
temperatures in this band, with the observed brightness temperatures being 
enhanced by almost 8 K in this region. \textsf{KCARTA} now includes a 
line-by-line NLTE model with Cousin linemixing in this region; in the future
we will be speeding up this code by using a lookup table of spectral and Planck
function modifiers, based on solar zenith angle.

We have also developed and/or interfaced packages to compute radiative 
transfer in the presence of scattering media such as clouds or aerosol layers.
Two well known scattering  algorithms have been interfaced with 
\textsf{KCARTA}. The first is \textsf{RTSPEC}, which uses a hybrid single 
scattering/Eddington approach \cite{dee:98}. Included in this package is 
code to compute Mie scattering tables for ice or water particle distributions,
as well as modifications to compute these tables for some of the aerosols in 
the \textsf{OPAC} database \cite{hes:98}. These tables drive all of the 
scattering computations of \textsf{KCARTA}. \textsf{RTSPEC} runs very rapidly,
but does not include the effects of (solar) beam scattering. By interfacing 
\textsf{DISORT} \cite{stam:88} with our package, we can include the effects of
solar beam scattering. However, this code does not run very fast, as it has to
perform a large number of matrix inversions at each spectral point. We have 
developed a rapid \textsf{kTWOSTREAM} scattering code that includes the 
effects of beam scattering. As has been done in the \textsf{CHARTS} 
\cite{mon:97} scattering code, \textsf{kTWOSTREAM} computes the reflection, 
transmission and layer emission of individual layers, adding them together to 
compute the overall scattering effects. \textsf{kTWOSTREAM} differs from 
\textsf{CHARTS} in that we only allow for two streams, and that we compute 
the radiative transfer individually on every spectral point, instead of 
building up a ``distribution'' to speed up the code.

Combining all the above features, we have a rapid and accurate code that 
computes the 
monochromatic atmospheric transmittances for different gas combinations, as
well as computing the monochromatic radiance. The whole package is called 
\textsf{kCARTA}, which stands for ``kCompressed Atmospheric Radiative 
Transfer Algorithm.''  This is an infrared ``monochromatic'' radiative 
transfer algorithm written for a one dimensional Earth atmosphere. 

We point out here that \textsf{kCARTA} should not be used for limb viewing, 
especially when using very high resolution instruments high in the
atmosphere. This is mainly a limitation for ballon borne instruments that can 
have very high spectral resolution.

\section{Non scattering Radiative transfer algorithm}

The standard Schwartschild equation for time independent radiation
transfer through a plane atmosphere, can be written as
\cite{goo:89,edw:92}
\begin{equation}
R(\nu) = R_{s}(\nu) + R_{layer emission}(\nu) + R_{th}(\nu) + R_{solar}(\nu)
\end{equation}
where the four terms are the surface, layer emissions, downward
thermal and solar respectively. Using an isotropic reflectance of
$1/2\pi$, and denoting the Planck function as $B(T)$, $\epsilon$ as
the surface emissivity, $T_{s}$ as the surface temperature, the
satellite viewing angle as $\theta_{satellite}$, the sun zenith angle
as $\theta_{solar}$, and discretizing the radiative transfer equation,
the above four terms are written out as
\begin{eqnarray*}
R_{s}(\nu) & = & \epsilon B(\nu,T_{s})
\tau_{1 \rightarrow \infty}(\nu,\theta_{satellite})
\end{eqnarray*}
\begin{eqnarray*}
R_{layer emission}(\nu) & = & \sum_{i=1}^{i=N} B(\nu,T_{i})
(\tau_{i+1 \rightarrow \infty}(\nu,\theta_{satellite})-
 \tau_{i \rightarrow \infty}(\nu,\theta_{satellite}))
\end{eqnarray*}
\begin{eqnarray*}
R_{th}(\nu) & = & \frac{1 - \epsilon}{\pi} \sum_{i=N}^{i=1} 
\int_{0}^{2\pi}d\phi 
\int_{0}^{\pi/2} d(cos\theta) cos\theta \times \\ 
& & B(\nu,T_{i})(\tau_{i-1 \rightarrow ground}(\nu,\theta)-
 \tau_{i \rightarrow ground}(\nu,\theta))
\end{eqnarray*}
\begin{eqnarray*}
R_{solar}(\nu) & = & \rho(\nu)
B(\nu,T_{solar})cos(\theta_{solar}) \times \\
& &                 \tau_{N \rightarrow ground}(\nu,\theta_{solar}))
                 \tau_{ground \rightarrow N}(\nu,\theta_{satellite}))
                 \Omega_{solar}
\end{eqnarray*}

where the solar reflectance $\rho(\nu)$ is either known or can be modelled 
in terms of the surface emissivity $\rho = \frac{1 - \epsilon}{\pi}$.
The above terms have been written in terms of layer to space transmittances. 
Alternatively the forward radiative transfer algorithm can easily be written 
iteratively; for example, the first two terms would be rewritten as : 
\begin{equation}
R(\nu) = \epsilon_{s}B(T_{s},\nu) \Pi_{i=1}^{i=N} \tau_{i}(\nu) + \\
         \sum_{i=1}^{i=N} B(T_{i},\nu) (1.0 - \tau_{i}(\nu))\\
         \Pi_{j=i+1}^{N} \tau_{j}(\nu)
\end{equation}
Looking at the second (summation) term, $(1.0 - \tau_{i}(\nu))$ is the 
emissivity of the layer while 
$(1.0 - \tau_{i}(\nu)) \Pi_{j=i+1}^{N} \tau_{j}(\nu)$ is the weighting 
function $W_{i}$ of the layer. Note that \textsf{kCARTA} has been written 
such that layer 1 is the lowest layer (highest average pressure) while layer 
100 is the highest layer (lowest average pressure). In what follows, the 
discretized form of the radiative transfer equation is used.

\subsection{Background thermal radiation}
The above expression for the background thermal radiation $R_{th}$ involves 
an angular integration. While the typical contribution due to this background
thermal can be small (less than 1 K), it is important to be able to 
accurately estimate and include this term. The actual integration over the 
half plane is of the form  
\begin{equation}
    \int_{0}^{2\pi}d\phi \int_{0}^{\pi/2} 
    d\theta \; sin\theta \; cos\theta \; B(T_{i}) 
    \tau_{i \rightarrow ground}(\nu,\theta)
\end{equation}
The mean value theorem can be used to rewrite this expression in terms of a 
single effective diffusive angle $\theta_{d}$ (along with the $2\pi$ factor 
that arises from the azimuthal integration) 
\begin{equation}
   \frac{1}{2} B(T_{i}) \tau_{i \rightarrow ground}(\theta_{d})
\end{equation}

This so-called diffusion approximation \cite{lio:80} reduces the computation 
for the downward thermal contribution to the form
\begin{equation}
    \frac{1}{2}B(T_{i}) \left[ \tau_{i-1 \rightarrow ground}
(\theta_{d1})- \tau_{i \rightarrow ground}(\theta_{d2}) \right] 
\end{equation}
where based on the layer to ground transmissions of the $i,i-1$ $th$
layers,  $\theta_{d1},\theta_{d2}$ are the optimum diffusion angles. 

The value of $\theta_{d}$ that is often used is that of $\arccos(3/5)$ 
\cite{lio:80}, especially for $k \le 1$.  A check of the the accuracy of 
using this angle at all layers and at all wavenumbers was carried out, using 
a selection of some of the AIRS regression profiles. Neglecting
solar radiation, the total radiation at the top of the atmosphere was
computed using the forward model and the reflected background thermal.
The truth for the background thermal was an angular integration over as many
as 40 Gaussian quadrature points in the $(0,\pi/2)$ interval, at each layer 
and at each wavenumber. The test value was obtained by using the diffusive 
angle of $\arccos (3/5)$ in the downward thermal at each layer. As expected,
in the wave number regions where the atmosphere was blacked out, the
diffusion approximation was perfectly acceptable. However where the atmosphere
was transparent, such as the 10 $\mu m$ window region, the errors could be 
larger than 0.2 K, especially if one used realistic land surface emissivity 
values of 0.8.

For the \textsf{AIRS} reference model, or any of the new generation of high 
resolution instruments, it is desirable to keep the brighness temperature 
errors $\le 0.1K$, throughout the wavenumber region encompassed by our 
spectroscopic \textsf{kCARTA} database. This meant that we needed to find a 
more accurate way of computing the background thermal radiation, as follows.
Instead of using the single fixed diffusive angle, an optimum diffusive 
angle was computed (as described below) at each layer and at each wavenumber. 
Rewriting the transmission as 
\[
\tau_{i \rightarrow ground}(\nu,\theta) = 
exp(-\sum_{j=1}^{j=i-1}k_{j}/cos\theta = exp(-k^{(i)}/cos\theta
\]
one can relate the angular integration for $R_{th}$ to the exponential 
integral of the third kind $E_{3}(k^{(i)})$, where 
$k^{(i)} = \sum_{j=i-1}^{j=1}k_{j}$.  This exponential integral can be 
evaluated (e.g. Numerical Recipes, MATLAB toolbox), and the optimum 
diffusion angle $\theta_{d}$ obtained from
\begin{equation}
         \theta_{d}(k^{(i)}) = \frac{- k^{(i)}}{ln (2 E_{3}(k^{(i)}))}
\end{equation}
In the limit
of $k^{(i)} \ll 1$, $\theta_{d} \rightarrow \arccos(0.5)$, while in the limit
of $k^{(i)} \gg 1$, $\theta_{d} \rightarrow \arccos(1.0)$. 

For a discrete set of values of $k^{(i)}$ between 0 and 10, the
diffusion angles were computed and saved. A polynomial fit to the
data, such that errors between the computed diffusion angle and the
polynomial approximations were always less than 0.5\%, was then made.
In this fashion, \textsf{kCARTA} can very quickly compute $\theta_{d}$
for an arbitrary $k^{(i)}$.

The accuracy of this computation was checked by propagating the
thermal background between the top of the atmosphere and the ground
using this polynomial approximation, and comparing it to the results
from the 40 point Gaussian quadrature. This was performed over the 605
to 2830 $cm^{-1}$ region, for a variety of AIRS regression profiles,
The typical brightness temperature error, computing an accurate diffusive 
angle at all layers and at all wavenumbers, was less than 0.001 K.

With the accuracy of the technique being successfully tested, a few 
variations on the technique were tried, to make the radiative transfer code 
run as quickly as possible, finally choosing the following. The 
$\tau(layer \rightarrow ground)$ factor at each level makes it apparent that 
the most significant contribution is from the bottom layers. For the topmost
layers ($100$ down to $J+1$), the simple diffusive approximation was
used (one angle, $\arccos(3/5)$ at all layers). For the bottom $J$
layers, the accurate diffusion angle for calculated for each layer,
based on the polynomial approximation to $\theta_{d}(k(i \rightarrow
ground))$. Assuming that the surface pressure is such that it occurs in one of
the lowest layers ($\leq$ 5), the value of J used produces less than 0.1 K 
errors in all profiles. For instance, where the atmosphere is blacked 
out (e.g. in the water region, about 1500 $cm^{-1}$), a value of $J=6$ was 
sufficient, while a transparent region such as about 2500 $cm^{-1}$, a larger 
value of $J(=30)$ was used. For the sampled profiles, using a surface 
emissivity value of 0.8, this always produced less than 0.1 K brightness 
temperature errors, and had the advantage of being faster than if the angle 
were computed at all layers, for all frequencies.

\subsection{Solar radiation}
The solar contribution is much easier to include than the thermal
contribution; assuming the sun radiates as a blackbody whose
temperature is 5600 K, the solar term that in incident at the earth's
surface is given by
\begin{equation}
 B(5600 K,\nu) \Omega_{solar} \tau(top \rightarrow ground) cos(\theta_{solar})
\end{equation}
where $\Omega_{solar} = \pi(r_{se}/r_{e})^{2}$ is the geometry factor
that accounts for the sun-earth distance and the $cos(\theta_{solar})$
is the geometry factor accounting for the solar radiation coming in at
an angle with respect to the vertical. This solar radiation is then
reflected back up to the instrument; either the surface reflectance $\rho$ is 
specified, or an isotropic reflectance factor of $(1-\epsilon)/\pi$ is used.

Both these thermal/solar radiation terms can be easily turned on/off
before runtime by simply setting relevant parameter switches. In
addition, the thermal contribution can be computed accurately, or by
using the upper level $arccos(3/5)$ approximation/accurate lower level
combination; similarly, we include the use of datafiles that give a more
correct solar beam spectrum incident at the TOA, instead of using 
5600 K throughout.

The code can also compute radiances for an upward looking instrument. Once 
again, this feature can be turned on/off by simply resetting appropriate 
parameters before runtime. Even at the lower layers, the default 100 layering
structure of \textsf{kCARTA} is probably too coarse for an accurate
estimate. This should not present problems, as the user can recompile 
and rerun the code, with a finer set of pressure layers in the lower
atmosphere. 

\subsection{Other features of the Clear Sky Forward Model}

When defining an atmosphere within which to compute radiances, {\sf kCARTA} 
allows the user to define the upper and lower pressure boundaries arbitrarily.
This flexibility allows the user to compute the radiance incident on a 
downward looking instrument that is on board an aircraft or at the top of the 
atmosphere, from a surface at sea level or at the top of a mountain. The 
downwelling solar and background thermal contributions are by default 
computed from the top of the atmosphere down to the surface. At the
surface, the contributions are appropriately weighted by the surface
reflectance, and then the complete upwelling radiance is computed from
the surface to the instrument.

While defining an atmosphere, the user refers to a set of ``mixed
paths'' which is simply a cumulative sum of gas optical depths,
weighted by a user specified amount. By using different mixed paths,
the user can easily make \textsf{kCARTA} compute radiances for different
atmospheres, for example ones in which the amount of water vapor is
slightly different.

In addition to the features mentioned above, the user can include files that 
allow the code to use a spectrally varying surface emissivity or a spectrally 
varying solar reflectivity, as well as change the surface and deep space 
temperatures. The user can also change the satellite viewing angle, as well 
as account for the changes of the local path angle due to the curvature of 
the earth.

\section{Jacobian algorithm}

Consider only the upward terms in the radiance equation (the layer
emission and the surface terms), reproduced here for convenience.
Assuming a nadir satellite viewing angle we have :
\begin{equation}
R(\nu) = \epsilon_{s}B(T_{s},\nu) \tau_{1 \rightarrow N}(\nu) +
\Sigma_{i=1}^{i=N} B(T_{i},\nu) (1.0 - \tau_{i}(\nu)) 
\tau_{i+1 \rightarrow N}(\nu)
\end{equation}

Differentiation with respect to the $m$-layer variable $s_{m}$, (which can be
gas amount or layer temperature $s_{m} = q_{m(g)},T_{m}$)
\[
\frac{\partial R(\nu)}{\partial s_{m}} = \epsilon_{s}B(T_{s}) 
\frac{\partial \tau_{1 \rightarrow N}(\nu)}{\partial s_{m}} +
\sum_{i=1} B(T_{i},\nu) (1.0 - \tau_{i}(\nu))
\frac{\partial \tau_{i+1}(\nu)}{\partial s_{m}} + 
\]
\begin{equation}
\sum_{i=1} \tau_{i+1}\frac{\partial B(T_{i},\nu) (1.0 - \tau_{i}(\nu))}
{\partial s_{m}}
\end{equation}

As usual, $\tau_{m}(\nu) = exp^{-k_{m}(\nu)}$,
$\tau_{m \rightarrow N}(\nu) = \Pi_{j=m}^{N} exp^{-k_{j}(\nu)}$. Performing the
above differentiation,  
\begin{eqnarray*}
\frac{\partial R(\nu)}{\partial s_{m}} & = &
\left[
\epsilon_{s}B(T_{s}) \tau_{1 \rightarrow N} \right]
(-1)\frac{\partial k_{m}(\nu)}{\partial s_{m}} + \\
& & \left[ \sum_{i=1}^{m-1}(1.0 - \tau_{i}(\nu)) B_{i}(\nu) 
\tau(\nu)_{i+1 \rightarrow N}
\right](-1)\frac{\partial k_{m}(\nu)}{\partial s_{m}} + \\  
& & \left[(1.0-\tau_{m}(\nu))\frac{\partial B_{m}(\nu)}{\partial s_{m}} -
B_{m}(\nu)\frac{\partial \tau_{m}(\nu)}{\partial s_{m}}
\right]\tau_{m+1 \rightarrow N}(\nu)
\end{eqnarray*}

The individual Jacobian terms in \textsf{kCARTA} code can then by
obtained as follows. Recall the layer transmission are related to
absorption coefficients by
\begin{equation}
\tau_{m}(q_{m(g)}) = exp^{-k(T_{m})q_{m(g)}/q^{ref(g)}_{m(g)}}
\end{equation}

Then for all gases other than water, using the SVD compressed notation,
\begin{equation}
k_{m(g)}(\nu) = \frac{q_{m(g)}}{q^{ref}_{m(g)}}
                \sum_{l=1}^{L} c_{l(g)}(T_{m},m) \Psi_{l}
\end{equation}
from which the gas amount derivative is simply 
\begin{equation}
\frac{\partial k_{m}}{\partial q_{m(g)}} = \frac{k_{m}}{q_{m(g)}}
\end{equation}

while for water, 
\begin{equation}
k_{m(w)}(\nu) = \sum_{l=1}^{L} c_{l(w)}(T_{m},m,q_{m}) \Psi_{l}
\end{equation}
from which the water amount derivative is 
\begin{equation}
\frac{\partial k_{m}}{\partial q_{m(w)}} = 
\sum_{l=1}^{L} \frac{\partial c_{l(w)}}{\partial q_{m(w)}} \Psi_{l}
\end{equation}

The temperature derivative can similarly be written as
\begin{equation}
\frac{\partial k_{m}}{\partial T_{m}} = 
\sum_{g=1}^{g=G} \sum_{l=1}^{L} 
\frac{\partial c_{l(g)}}{\partial T_{m}} \Psi_{l}
\end{equation}
where the double sum is over the singular vectors and the gases.

While doing the spline interpolations of the coefficients $c_{l(g)}$,
the derivatives $\frac{\partial c_{l(g)}}{\partial T_{m}}$,
$\frac{\partial c_{l(w)}}{\partial q_{m(w)}}$ can be obtained
concurrently \cite{wil:89} in the compressed space.  (These
Jacobians can also be calculated, in compressed space, but by
``perturbing'' the gas amounts/layer temperatures and then doing a
finite difference derivative, before performing the uncompression).
Multiplying by the orthonormal basis matrix $U$ then immediately gives
the analytic derivatives.  Performing the calculations of the
Jacobians in the compressed representation is therefore easily
achieved. As these radiance Jacobians are obtained in 25 $cm^{-1}$
chunks, for all 100 layers, they are easier to obtain than finite
difference Jacobians.  

The solar and background thermal terms for inclusion in the Jacobian
calculations are also included in the algorithm. However, due to the
increase in run-time of the code when computing the Jacobians, the
only possible computation for the thermal background Jacobians is
using the diffusive approximation $\arccos(3/5)$ at $all$ levels,
independent of whether the forward model radiative transfer algorithm
used the accurate computation or the diffusive/accurate combination.
Because of this, there would be slight differences if one compared the
computed Jacobians to those obtained using finite differences between
two almost similar parameterizations of the forward model.

The Jacobians obtained using the compressed representation are much
faster than uncompressing the coefficients, doing a radiative
transfer, perturbing the relevant layer, and doing another radiative
transfer, after which a finite difference radiance Jacobian is
obtained. The reason is easy to see -- one would have to do these
perturbed calculations for $each$ gas amount and or temperature, 
at $each$ layer, instead of obtaining the Jacobians in big chunks.

In addition to the gas amount/layer temperature Jacobians described
above, the Jacobians with respect to the surface temperature and
surface emissivity are also computed. The Jacobian of the background
thermal contribution with respect to the surface emissivity, and the
Jacobian of the solar contribution with respect to the surface
emissivity are also output.  Additionally, weighting functions
$W_{i}(\nu)$ are also computed and output as part of the overall
Jacobian file :
\begin{eqnarray*}
R_{layer emission} (\nu) &  = &
\Sigma_{i=1}^{i=N} B(T_{i},\nu) (1.0 - \tau_{i}(\nu)) 
\tau_{i+1\rightarrow N}(\nu) \\
& = &\Sigma_{i=1}^{i=N} B(T_{i},\nu) W_{i}(\nu)
\end{eqnarray*}

Another feature of the code is that the Jacobians can be output in any
of three modes. The first is a raw $d(radiance)/d(variable)$ mode,
where $variable$ could be gas amount, layer temperature etc. Another
mode is a $\delta(variable \times d(radiance)/d(variable))$ mode, where if
$variable$ is a gas amount, then we have appropriately weighted the
Jacobian with the gas amount at that layer. The third mode is 
$\delta(variable) \times d(brightness temperature)/d(variable)$ mode, where
all the results now are Jacobians with respect to brightness
temperatures.

Once again, the turning on or off of the Jacobians can be achieved
simply by setting the appropriate parameter at run time. Furthermore,
the inclusion of thermal background to the Jacobian can be turned off
(resulting in a significant decrease in run time) at the expense of
incorrectly estimating the Jacobians at the lowest levels (as these
are where the bulk of the background thermal contribution comes from).
The temperature Jacobians can include only the Planck term, or the weighting
function term, or both (see Case Study 2 later). 

\section{Case Study : Contributions of various terms to temperature 
Jacobians}

If one removes the surface, solar and background thermal contributions
from the radiance that one measures at the top of the atmosphere, then
only the layer emission terms contribute :
\begin{equation}
R(\nu) = \Sigma_{i=1}^{i=N} B(T_{i},\nu) (1.0 - \tau_{i}(\nu)) 
\tau_{i+1\rightarrow N}(\nu) = \Sigma_{i=1}^{i=N} B(T_{i},\nu) W_{i}(T_{i})
\end{equation}
where we have rewritten the radiance in terms of the weighting functions $W$

If we compute the radiance using the above expression, errors in the layer 
temperature would manifest themselves in one of two places : in the Planck
term, or in the transmission terms. An investigation of which of these two 
terms dominates would tell us what is required more accurately; the layer 
temperature or the temperature dependance of the spectroscopy. In other words,
differentiating the above equation with respect to the temperature of the 
$m$-th layer, we can separate the resulting Jacobian into the sum of two
contributions :
\begin{eqnarray*}
\frac{\partial R(\nu)}{\partial T_{m}} & =  &
\left[ \sum_{i=1}^{m-1} B(T_{i},\nu) 
      \left\{ \frac{\partial \tau_{i+1 \rightarrow N }(\nu)}{\partial T_{m}} -
      \frac{\partial \tau_{i \rightarrow N}(\nu)}{\partial T_{m}} \right\}- 
\right. \\
 & &  \left .B(T_{m},\nu)\tau_{m+1\rightarrow N}(\nu)  
\frac{\partial \tau_{m}(\nu)}{\partial T_{m}} \right]+ \\
& & \left[ \frac{\partial B(T_{m},\nu)}{\partial T_{m}}
(1.0-\tau_{m}(\nu))\tau_{m+1 \rightarrow N}(\nu)\right] \\
 & =  & \frac{\partial W_{m}}{\partial T_{m}} \; B(T_{m}) +
     \frac{\partial B(T_{m})}{\partial T_{m}} \; W_{m}(T_{m})
\end{eqnarray*}
where the first contribution is that due to the transmission terms
changing because of the temperature change, while the second is that
of the Planck term changing because of the temperature change. 

To compute these terms, the Jacobian subroutines include a runtime switch to 
compute either the weighting function dependance
$\partial W_{m}/\partial T \; B_{m}$ terms, the Planck term dependance 
$\partial B(T)/\partial T \; W_{m}$, or both. To make the comparisons more 
meaningful, the surface emissivity was set at $1.0e^{-5}$ and background 
thermal and solar set to 0.0, so that all terms involving the 
surface emissivity in the radiance calculation were turned off.
As above, these two changes were documented for various US Standard
Profiles, and convolved over the AIRS and IASI instrument functions,
to compare the results.

Before presenting the results, an estimate of the relative sizes is
made.  The Planck radiance, as usual, is given by
\begin{equation} 
B(\nu,T)=\frac{c_{1} \nu^{3}}{exp^{c_{2} \nu/T}-1}
\end{equation} 
where $c_1,c_2$ are the radiative constants, $T$ is the temperature and 
$\nu$ is the wavenumber (in $cm^{-1}$). This expression is easily 
differentiated with respect to temperature, giving estimates for 
$B, \partial B/\partial T$. As is easily verified, for a typical temperature 
of $T \simeq 300 K$, the Planck radiance peaks at about 600 $cm^{-1}$, 
attaining a value of
0.154 Watts m-2 sr-1/ cm-1. On the other hand, for the same
temperature, the derivative peaks at about 800 $cm^{-1}$, with a value
of about $1.75 \times 10^{-3}$ Watts m-2 sr-1/ cm-1 K-1. In the
605-2830 $cm^{-1}$ wavenumber region, the radiance is about 100 times
larger than the derivative in the small wavenumber region, decreasing
to about a factor of 20 times larger in the other end of the infrared
spectrum (2800 $cm^{-1}$ region). Using a typical peak value of a
weighting function as about 0.1, this means that $ \delta B(T) \;
W_{m} \simeq 10^{-4}$.

An estimate of the size of the other term is not so simple, but we
proceed as follows, following Liou \cite{lio:80} and Edwards
\cite{edw:92}.  The transmission (or optical depth) is related to the
monochromatic absorption coefficients by $k(\nu) = exp^{-K(\nu)}$.
In general, for a line whose center frequency is $\nu_{0}$,
$K(\nu,\nu_{0}) = q S(T,\nu_{0}) g(\nu,\nu_{0})$ where $q$ is the gas amount,
$S$ is the line strength adjusted for temperature and $g$ the normalized 
line shape function, which is Lorentz at the lower levels in the atmosphere 
and Doppler at the higher levels.

For an arbitrary temperature $T$, the line strength can be calculated
relative to a reference line strength $S(T_{ref})$
\cite{lio:92,edw:92,goo:89}
\begin{eqnarray*}
S(T) & = & S(T_{ref}) \frac{Q(T_{ref})}{Q(T)}
\frac{exp^{-hcE_{l}/kT}}{exp^{-hcE_{l}/KT_{ref}}} \;
\frac{1-exp^{-hc\nu_{0}/kT}}{1-exp^{-hc\nu_{0}/kT_{ref}}}
\end{eqnarray*}
where $Q$ is the partition function, $E_{l}$ is the energy of the
lower level of the transition and $\nu_{0}$ is the center frequency of
the transition. The expression $exp^{-hcE_{l}/kT}/Q(T)$ is
simply the fractional number of molecules in the lower energy state
$n(E_{1})/n$.  For atmospheric conditions, only the rotation
partition function has a significant temperature dependence, varying
as $T^{n}$ \cite{lio:80} where $n=1$ for linear molecules, and $n=3/2$ for
nonlinear molecules. We can thus rewrite the expression for
the line intensity strength as 
\begin{equation}
S(T) \simeq S(T_{ref}) \left( \frac{T_{ref}}{T} \right)^{n}
\frac{exp^{-hcE_{l}/kT}}{exp^{-hcE_{l}/KT_{ref}}} 
\frac{1-exp^{-hc\nu_{0}/kT}}{1-exp^{-hc\nu_{0}/kT_{ref}}}
\end{equation}

With this as a simple model, we can write the absorption coefficient
as $K = q S_{0} (1/T)^{3/2} exp^{-hcE_{l}/KT} (1-exp^{-hc\nu_{0}/kT}) $. 
Here $S_{0}$ is the line strength, along with other factors such as the 
linewidth term. This clearly implies that $\partial K/\partial T \simeq K/T 
\times factor$; thus as one goes higher in the atmosphere, the gas amount
decrease would lead one to expect $\partial K/\partial T$ to decrease.

By putting in appropriate values of
$S$ and $E_{l}$, such as $E_{l} \simeq 250 k$ and $S \simeq 50,50000$
for a weak line and a strong line respectively, one obtains $\partial
\tau/\partial T ~ \simeq 5 \times 10^{-4}$ for transmissions of about
0.5.  A quick check of this number can easily be made using ${\sf
  kCARTA}$ -- using the US Standard Profile, layer to space
transmissions were computed in the wavenumber region 1055-1080
$cm^{-1}$, and then recomputed for the same profile, but with a 0.1 K
temperature offset. The finite difference temperature derivatives were
on the order of $2 \times 10^{-3}$. Thus using typical Planckian
radiance of 0.1, and the transmission derivative estimate of about
0.001, means that $\delta W_{m} \; B(T) \simeq 10^{-4}$, which is the
same order of magnitude of the $\delta B(T) \; W_{m}$ estimate.

To estimate the size of the $\partial B$ terms compared to the $\partial W$ 
terms, we can proceed as follows. For any layer we compare
\begin{eqnarray*}
\frac{\partial B_{m}}{\partial T_{m}} W_{m} \;\;\;   & :\;\;\;  &
\frac{\partial W_{m}}{\partial T_{m}} B_{m} \\
\end{eqnarray*}

Dividing both sides by $B_{m}W_{m}$ and relating $W_{m}$ to the transmissions
and hence absoprtion coefficients $K_{m}$, we get 
\begin{eqnarray*}
%\frac{\partial B_{m}}{\partial T_{m}} \frac{m}{B_{m}} \;\;\;   & :\;\;\;  &
%\frac{\partial K_{m}}{\partial T_{m}} 
%\frac{\tau_{m \rightarrow N}}{\tau_{2 \rightarrow N}} \frac{m}{m-\tau_{m}}\\
\frac{\partial B_{m}}{\partial T_{m}} \frac{1}{B_{m}} \;\;\;   & : \;\;\;  &
\frac{\partial K_{m}}{\partial T_{m}} \frac{\tau_{m}}{1-\tau_{m}}\\
\end{eqnarray*}

Consider the layer closest to the ground (which is one in \textsf{kCARTA}).
If the wavenumber region is such that the  absorption coefficients are
large ($K_{1} \gg 1$) then the $\partial B_{1}/\partial T$ term dominates, 
as $\tau_{1} \simeq exp^{-K_{1}}$. On the other hand, if the wavenumber 
region of interest is almost transparent, then $K_{1} \ll 1$, and this makes 
the two terms comparable in size.

Now consider layers higher up in the atmosphere. Remebering that the gas 
amounts fall off with height, and that the $\partial K/\partial T$ depends on
$K$, this term rapidly becomes smaller than $\partial B_{m}/\partial T$, 
leading one to expect the $\partial B_{m}/\partial T$ term to dominate.

These estimates were vindicated by the \textsf{kCARTA} runs. In the lowest 
levels of the atmosphere, the two terms were of similar magnitude in the 
window regions, while though small in magnitude, the planck term was larger 
in the ``blacked out'' regions. For both regions, the 
$\delta B(T) \; W_{m}$ began to dominate in the troposphere and above. The 
following figures illustrate the two terms convolved over the AIRS spectral 
response function, for the US Standard Profile. The first set of plots is in 
the 900 \wn window region, while the second set is in the 1500 \wn water 
region.

%%these files are in /salsify/scratch4/Sergio/
%%radiances are in DB_TERMS     saved stuff from ../DK_TERMS/testdkh1JACCON
%%all saved in /DB_TERMS/dbdk.mat
%% plotjac(0,-1,20,90,100,1500,1605,db,raf,dk,raf)    
%% plotjac(0,-1,20,90,100,680,880,db,raf,dk,raf)    

%%>> ind=1:1000:60000; 
%%>> jacT2 = jac2(ind,97*1 + (1:97)); 
%%>> jacT1 = jac1(ind,97*1 + (1:97)); 
%%>> dbt2=dBTdr(w(ind),rad2(ind))*ones(1,97);
%%>> dbt1=dBTdr(w(ind),rad1(ind))*ones(1,97);
%%>> dbt2=dbt2.*jacT2;                       
%%>> dbt1=dbt1.*jacT1;
%%>> figure(2); pcolor(w(ind),lay,dbt2(:,lay)'); shading('interp'); 
%%>> print -depsc2 planck_1405.eps
%%>> caxis([-0.05 0.15]); colorbar; title('planck');
%%>> figure(1); pcolor(w(ind),lay,dbt1(:,lay)'); shading('interp'); 
%%>> caxis([-0.05 0.15]); colorbar; title('tau');   
%%>> print -depsc2 tau_1405.eps   

%%\begin{figure}
%%\includegraphics[width=5.5in]{/salsify/scratch4/Sergio/DB_TERMS/water.eps}
%%\caption{(top)d(Planck)/dT AIRS Temperature Jacobians 
%%         (bottom)d($\tau$)/dT AIRS Temperature Jacobians}
%%\label{sampleplot}
%%\end{figure}

%%july 2001
%\begin{figure}
%\includegraphics[width=6.0in]{EPS_FILES/temper_905.eps}
%\caption{(top)   d($\tau$)/dT AIRS Temperature Jacobians
%         (bottom)d(Planck)/dT AIRS Temperature Jacobians} 
%\label{temp_905}
%\end{figure}

%\begin{figure}
%\includegraphics[width=6.0in]{EPS_FILES/temper_1405.eps}
%\caption{(top)   d($\tau$)/dT AIRS Temperature Jacobians
%         (bottom)d(Planck)/dT AIRS Temperature Jacobians} 
%\label{temp_1405}
%\end{figure}

%%\begin{figure}
%%\includegraphics[width=5.5in]{/salsify/scratch4/Sergio/DB_TERMS/co2.eps}
%%\caption{(top)d(Planck)/dT AIRS Temperature Jacobians 
%%          (bottom)d($\tau$)/dT AIRS Temperature Jacobians}
%%\label{sampleplot}
%%\end{figure}

\section{Conclusions}
The \textsf{kCARTA} package offers the user many desirable features. When used
to compute optical depths for an arbitrary Earth atmosphere, it is much faster
than line-by-line codes, and the accuracy of its spectroscopic database has 
been extensively compared  against \textsf{GENLN2}. The computed clear sky 
radiances include an accurate estimate of the background thermal. 

\section{Acknowledgements}
This work was supported in part by NASA grant number 05-5-28045. We wish to 
thank users of kCARTA that have provided us with feedback. Ji Gou of UMBC 
performed multiple runs of \kc, to assess the different scattering codes. 
Dave Tobin of UW-Madison was instrumental in helping develop the \cd 
linemixing code, as well as modifying the water continuum coefficients. Dave 
Edwards of NCAR provided the {\sf GENLN2} line-by-line code to test the 
{\sf kCompressed} database against, allowed us to use modifications of some of 
his subroutines in {\sf KCARTA}, as well as helped us debug the NLTE code. 
In addition we thank Pat Arnott of the 
Desert Research Institute for motivating discussions about the applicability 
of a single scattering versus twostream approach, Frank Evans of the 
University of Colorado for help in getting RTSPEC interfaced, and Istvan 
Laszlo of University of Maryland, College Park for help in interfacing 
DISORT.

\bibliographystyle{unsrt}
\bibliography{/home/sergio/PAPERS/BIB/atmspec2002}

\end{document}

